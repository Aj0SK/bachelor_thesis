\chapter{Method adjustments}

\label{kap:methAdjust} % id kapitoly pre prikaz ref

In this chapter we will look at the main challenges of our method and the ways
we can overcome them.

\section{Experiment data}

Our data comes from two organisms, one is Saprochaete ingens. We use this organism
as a positive example which we will want to sequence. The second organism is
Saprochaete fungicola. Both of these organisms are yeasts, not sharing the large
parts of the DNA. We choose these organisms as they were already sequenced by our
department and we have all the data from the sequencing easily available.

\section{Oscillations}

We already saw in TODO that one of the biggest differences between simulated and
real signal is the noise which can be seen as small oscillations of the signal. This is
one of the reasons that we use windows - so small oscillations within the window
are dealt with. However, when our method is used without any modifications it
produces much longer level string for real signal than for simulated signal.
The problem arises when the oscillations are located on the borders of the windows.
There are several possibilities to deal with this behavior not affecting speed too
much. We tried several smoothing algorithms to reduce noise. From simple as the
moving average and the moving median to some more complicated as a kalman filter.
For now, we will stick to the faster solutions and choose one of the simpler techniques.

\section{Continuity of signal}

The next difference that was visible from the comparison of real signal level string
and simulated signal level string is that the real signal is somewhat continuos instead
of a simulated signal that more jumps.

The another solution that can be combined with the previous solutions is only.

