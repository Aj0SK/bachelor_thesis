\chapter{Testing}

\label{kap:testing} % id kapitoly pre prikaz ref

In this chapter, we will try to show how we are going to test our proposed method.
This includes finding some baseline. The method that is different from our proposed
algorithm and that reasonably solves the same problem using other techniques.

\section{Machine learning baseline}

As there are numerous patterns in the signal coming from DNA and a whole sequencing
process is not that well defined, we chose to apply machine learning methods to
tackle this problem. We decided to use methods from the area of supervised learning
as we can produce a lot of labeled data with ease. We now create different scenarios
that we will use for training and testing and formulate them as a classification task.

\subsection{Test scenarios}

We will be training and testing this model in the following scenarios:

\paragraph{Scenario A} In this scenario we want to distinguish between mitochondrial
and nuclear DNA of one organism. Mitochondrial DNA is a special type of DNA and
contains special patterns. It has also some obvious patterns as a higher ratio of
certain nucleotides.

\paragraph{Scenario B} In this scenario we want to correctly classify from which
contig or chromosome of our organism the input signal comes. We expect that this
will be harder based on the fact that chromosomes can share some not that small parts.

\paragraph{Scenario C} In this scenario we want to classify DNA sequence of two
different organisms.

As we already mentioned, all these scenarios are classification problems so we have
to formulate them in this way. In all 3 scenarios, we will have the input signal as a
vector of floating-point numbers. In each case, we want to predict output as coming
from one of the classes. In scenarios A and C we have 2 classes. In scenario C we
will have 6 categories.

\subsection{Methods}

Our tested methods were from the area of supervised learning.

\subsubsection{Convolutional Neural Networks}

As there are hints that CNN are well-performing against problems that can be
represented by an image, we got an idea that they could be useful in our scenario.
This was motivated by a realization that experts can extract some information from
the pure image of the signal.

The structure of our tested CNN was typically very similar and consisted of these
layers:

\paragraph{Input layer}
We decided to make our input layer also convolutional. We had an idea to reverse
every second line to have subsequent signals as close as possible. This was later
proved to be negligible.

\paragraph{Convolutional layers}
This part consists of several repeating parts. There are 3 layers used in
this part of the model: convolutional layer, activation layer, and pooling layer.
After every convolutional layer, there is an activation layer. Convolutional layer uses
the window of size $3x3$ as it is good to have the window of odd dimensions.

\paragraph{Flatenning layer}
This layer flattens multidimensional input into one dimensional and is only technicality.

\paragraph{Fully connected layer}

In our model, this part of the network consisted of only 2 layers.

Overall, our model consisted of about 15 layers and about 3 dropout layers that
helped us prevent overfitting.

\subsubsection{Support Vector Machines}

Another model that we used were SVM. We tried this model as SVM are consider good
classifiers. The kernels that we tried are RBF and Linear kernel.

\subsection{Datasets}

Our data comes from two organisms, one is Saprochaete ingens. This is an organism
on which DNA we will train our models for scenario A and scenario B. Second organism,
for use in Problem C is organism Saprochaete fungicola.

In order to prepare our datasets, we had to resolve some problems. One of the
problems that arose was that the number of samples from non-mitochondrial DNA
outnumbers samples from mitochondrial DNA. This would cause some bad hypothesis
as predicting always majority element to get very high accuracy. This forced us
to tweak the dataset in a way, that every class is in every problem equally represented.

Both of the methods we wanted to use need or work well for a fixed size input.
As the signal from the DNA can have very different length and the same length is
very uncommon, we need a way to resize this signal to some fixed size. It’s also
good for us to predict the class of signal from only a smaller portion of the signal.
This makes the idea of cutting the signal into fixed-size windows very convenient
as the signal of the one whole read can be really long. This also brings the question
of what is the ideal size of the window.

Using non-overlapping windows brings us the advantage that we don't have to bother
with leaking some part of training information into testing. Also, as we cut input
signals into some fixed-size, non-overlapping windows, we got rid of mean and
standard deviation in all windows.

\subsection{Results of our models}

We bring results of our testing in the table \ref{tab:mlResults}. 

\begin{table}
\caption[Accuracy of our model during testing]{Accuracy of our model during testing}
%id tabulky
\label{tab:mlResults}
\begin{center}
\begin{tabular}{|l|l|l|}
\hline
Scenario & Method & Testing accuracy (in \%)\\
\hline
A & CNN & 90 (baseline 50)\\
A & SVM & 54 (baseline 50)\\
B & CNN & 30 (baseline 16)\\
B & SVM & 18 (baseline 16)\\
C & CNN & 55 (baseline 50)\\
\hline
\end{tabular}
\end{center}
\end{table}

As we expected, our methods were most successful in classifying scenario A, determining
nuclear-DNA vs mitochondrial-DNA. All the support vector machines were a disappointment.
They were very slow during the process of learning so they couldn't train on a lot of
data. Even after considerable time, they were very close to random classification.
What we learned after reading documentation is that a lot of popular libraries
implementing SVM work only on CPU and are not GPU accelerated which can cause
problems for bigger data sets. Even the linear kernel showed worse performance than the RBF kernel.

\medskip

Scenario B seems to be considerably more difficult as scenario A as CNN could obtain
only 30\% accuracy. As the number of classes in our dataset (Saprochaete ingens)
is 6. 30\% is some improvement over $(100/6)=16\%$.

From the overall process of training, we gained confidence, that CNN could
be used to help with these problems after some time invested in looking for
problems of this model.

\medskip

In scenario C, CNNs didn't work well at all. They didn't obtain any ability to predict
the class even after adding more hidden layers and making the network more complex.

\medskip

First estimate was to try a window of size 120. This proved to be quite a good estimate.
For CNN we favored if the size of the window was square. The size of the good window
proved to be around $14\cdot 14$.
