\chapter*{Introduction} % chapter* je necislovana kapitola
\addcontentsline{toc}{chapter}{Introduction} % rucne pridanie do obsahu
\markboth{Introduction}{Introduction} % vyriesenie hlaviciek

The DNA is one of the most important molecules in the living organisms. It consists
of the nucleotides A, C, G, T. The MinION sequencer is one of the tools that enable
us to read this DNA so we can study numerous secrets of the DNA molecule and its
implications on our lives. The MinION sequencer however does not produce the DNA
sequence directly. Instead, it produces noisy signal that was created when the DNA
molecule passed through one of the MinIONs nanopores. Due to the nature of the
nanopores, we are able to reconstruct the DNA sequence from this noisy signal
using the using the process called base-calling. The another advantage of the MinION
sequencer is that it is able to throw back the DNA fragment that is currently passing
through one of his nanopores. The selective sequencing is the idea, that based on the
signal currently produced by the MinION, we decide if we want to continue in sequencing
this DNA fragment. This enables us to process more of a fragments, we are interested in.
To this day, the base-calling algorithms are too slow for us, to turn signal into DNA
sequence and decide afterwards. During the time, we are sequencing the signal we obtained
from the beginning of the DNA fragment, the fragment could already passed the phase of
sequencing. This is why, we have to work with the raw signal. This is quite hard as the
signal is very noisy. We propose a method that discretize a raw signal and then subsequently
tries to find it in a longer reference discretized signal. The problem of this method is, that
most of the time we are only given the reference in the form of the DNA so we need to create a simulated signal.

In the first chapter, we are looking at the necessary background to understand what
are the main problems in this area and what are the solutions that already exist.

In the second chapter, we propose a discretizing method and method for simulating the signal.
We then look at the different aspects of the real and simulated signal and we try several
methods to make them have the similar properties. At the end of the second chapter,
we show the summary results of our effort.

In the third chapter, we are looking at the aspects of our discretizing method and
its usage in building the index data structure that is able to respond to fast
request of finding if the query signal is present in the index.

