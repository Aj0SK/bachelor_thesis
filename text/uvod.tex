\chapter*{Introduction} % chapter* je necislovana kapitola
\addcontentsline{toc}{chapter}{Introduction} % rucne pridanie do obsahu
\markboth{Introduction}{Introduction} % vyriesenie hlaviciek

The sequence of DNA is one of the most important molecules in the living organisms. It consists
of the nucleotides adenine (A), cytosine (C), guanine (G), and thymine (T).
The MinION sequencer enables us to read this DNA so we can study numerous secrets
of the DNA molecule and its
implications in our lives. The MinION sequencer, however, does not produce the DNA
sequence directly. Instead, it produces a noisy signal created as the DNA
passes through one of the MinIONs nanopores. Due to the nature of the
nanopores, we are able to reconstruct the DNA sequence from this noisy signal
using the process called base-calling.

Another advantage of the MinION sequencer is that it is able to release the DNA fragment that is currently passing
through one of his nanopores without further sequencing. Selective sequencing is an idea that based on the
signal currently produced by the MinION, we can decide if we continue the sequencing
or release the DNA fragment. Selective sequencing enables us to enrich sequencing for the
fragments, we are interested in.

To this day, base-calling algorithms are too slow to apply this idea in practice.
During the time, we are base-calling the signal obtained
from the beginning of the DNA fragment, the fragment could have already passed
through the pore and could been already sequenced in its entirity.

This is why, we have to work with the raw signal. This is quite hard as the
signal is very noisy. We propose a method to discretize the raw signal and then subsequently
attempt to find it in a longer reference discretized signal. The problem of this method is that
most of the time we are only given the reference in the form of the DNA so we need to create a simulated signal.

In the first chapter, we present the necessary background needed to understand
the main challenges in this area and existing solutions.

In the second chapter, we propose a discretization method and a method for simulating the signal
from the reference sequence. We look at different properties of the real and simulated signal and we test several
methods to make them to have similar properties. At the end of the second chapter,
we summarize experimental results of our efforts in this area.

In the third chapter, we use the methods developed and tested for signal discretization
to build the index data structure that is able to respond fast to
requests whether the query signal is present in the index.

