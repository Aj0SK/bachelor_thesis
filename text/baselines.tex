\chapter{Baselines}

\label{kap:baseline} % id kapitoly pre prikaz ref

In this chapter, we will try to find baseline. Method that is different from our
proposed algorithm and that solves the same problem using other techniques. This
will be used later as a comparison to our method.

\section{Baseline using machine learning methods}

As there are numerous patterns in the signal coming from DNA and also this process
is quite erroneous, we chose to apply machine learning methods to tackle this problem.
We decided to use methods from area of supervised learning as we can produce a lot
of labeled data with ease. We now formulate different scenarios that we will be working
on and formulate them as a classification task.

\subsection{Test scenarios}

We will be training and testing this model in the following scenarios:

\paragraph{Scenario A} In this scenario we want to distinguish between mitochondrial and nuclear DNA
of a same organism. Mitochondrial DNA is a special type of DNA and contains special
patterns. It also has some obvious patterns as higher number of certain nucleotides.

\paragraph{Scenario B} In this scenario we want to correctly classify from which
contig or chromosome of our species the input signal comes. We expect that this
will be harder based on fact that chromosomes can share some not that small parts.

\paragraph{Scenario C} In this scenario we want to classify two different organisms.

As we already mentioned, all these scenarios are classification problems so we have
to formulate them in this way. In all 3 scenarios we will have input signal as a
vector of floating point numbers. In each case, we want to predict output as coming
from one of the classes. In Scenarios A and B we have 2 classes.

\subsection{Methods}

Our tested methods were from the area of supervised learning.

\subsubsection{Convolutional Neural Networks}

As there are hints that CNN are well-performing against problems that can be
represented by an image, we got an idea that they could be useful in our scenario.
This was motivated by a realization that experts can extract some information from
the pure image of the signal.

The structure of our tested CNN was typically very similar and consisted of these
layers:

\paragraph{Input layer}
We decided to make our input layer also convolutional. We had an idea to reverse
every second line to have subsequent signals as close as possible. This was later
proved to be negligible.

\paragraph{Convolutional layers}
This part consists of several repeating parts. There are 3 basic layers used in
this part of the model: Convolutional layer, activation layer, and pooling layer.
After every convolutional layer there is activation layer. Convolutional layer uses
window of size $3x3$ as it is good to have window of odd dimensions.

\paragraph{Flatenning layer}
This layer flattens multidimensional input into one dimensional and is only technicality.

\paragraph{Fully connected layer}

In our model, this part of the network consisted of only 2 layers.

\medskip

Overall, our model consisted of about 15 layers.

\subsubsection{Support Vector Machines}

Another model that we used were SVM. We tried this model as SVM are consider good
classifiers. The kernels that we tried are RBF and Linear kernel.

\subsection{Datasets}

Our data comes from two organisms, one is Saprochaete ingens. This is an organism
on which DNA we will train our models for problem A and problem B. Second organism,
for use in Problem C is organism Saprochaete fungicola.

In order to prepare our datasets, we had to resolve some problems. One of the
problems that arose was that the number of samples from non-mitochondrial DNA
outnumbers samples from mitochondrial DNA. This would cause some bad hypothesis
as predicting always majority element to get very high accuracy. This forced us
to tweak the dataset in a way, that every class is in every problem equally represented.

Both of methods we wanted to use work well or need a fixed size input. As the signal
from the DNA can have very different length and the same length is very uncommon, we
need a way to resize this signal to some fixed size. It’s also good for us to obtain
the class of signal from only a smaller portion of the signal. This makes the idea
of cutting the signal into fixed-size windows very convenient as the signal of the
whole 1 read can be really long. This also brings the question of what is the ideal
size of the window.

Also, as we cut input signals into some fixed-size, non-overlapping windows, we got
rid of mean and standard deviation in all windows.

\section{Results of our models}

As we expected, our methods were most successful in classifying Problem A, determining
nuclear-DNA vs mitochondrial-DNA. All the support vector machines were disappointment.
They were very slow during the process of learning so they couldn't train on a lot of
data. Even after considerable time, they were very close to random classification.
What we learned after reading documentation is that popular library scikit-learn
works only on CPU and is not accelerated which can cause problems for bigger data
sets. Even linear kernel showed worse performance than RBF kernel.

\medskip

Problem B seems to be considerably more difficult as problem A as CNN could obtain
only 30\% accuracy. As the number of classes in our dataset (Saprochaete ingens)
is 6. 30\% is some improvement over $(100/6)=16\%$.

From the overall process of training, we gained some confidence, that CNN could
be used to help with these problems after some time invested into looking for
problems of this model.

\medskip

In the Problem C, CNNs crashed hard. They didn't obtain any ability to predict
class even after adding some more hidden layers.

\medskip

Regarding windows size, the advice from my mentor was to try window of size 120.
This proved to be quite a good estimate but for CNN I saw as better if the size
is square of some number. The sizes of good window proved to be around $14\cdot 14$.
