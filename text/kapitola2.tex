\chapter{Squiggle Discretization and Similarity Between Squiggles}

\label{kap:proposedMethod} % id kapitoly pre prikaz ref

In this chapter, we will present our proposed method that will allow us to more
easily work with the squiggles.

\section{Squiggles as Sequencing of Discrete Observations}

When we obtain the simulated signal from the reference DNA, we want to be able
to decide during the process of sequencing if the DNA molecule currently passing
through the pore is from this reference signal. This comes down to finding if the
current squiggle is present in this reference signal. If both of these signals would
be short, it would be quite possible to do even manually. As we will see, to a
certain extent, we can see that the two signals are similar. However, we need to
do this fast and on very long signals.

The idea of our solution is to discretize the simulated signal of the
reference sequence. We want to come with some better representation of the
squiggles that represents them well and also provide some advantages such as
easy searching in this representation. In our current setup, we are
not very limited by the preprocessing time but we need to decide very fast once
we are sequencing the DNA. This puts some speed limitations on our discretizing
process during the processing of the DNA. What we will do is that we will
represent the signal as a string. We split the signal vertically into several
windows in such a manner that everything between some minimal and maximal
value is in some window. These windows are of constant length and do not
overlap. More formally, let $a_i$ be the signal at the time $i$, $m$ the number of
vertical windows. Now, we have some signal $s$ of length $n$, $s=a_1a_2a_3\cdots a_n$.
Besides this, we have some values $min_a$, $max_a$ that $\forall a_i: min_a \leq a_i < max_a$.
We say that signal $a_i$ is in $j$-th window if it holds that:

\begin{center}
$min_a + j\cdot \frac{max_a-min_a}{m} \leq a_i < min_a + (j+1)\cdot \frac{max_a-min_a}{m}$
\end{center}

If the signals $a_0, a_1, \cdots, a_n$ are in the windows $w_0, w_1, \cdots ,w_n$
respectively, we call $ls$ such that $ls=w_1w_2\cdots w_n$ the level string of the
signal $s$. After this, we will contract subsequent occurrences of the same level.
So level string "aabccaa" becomes "abca". An example of this method on some signal
for $w=6$ is on TODO. 

TODO: Add image of short raw signal with the levels and level string

This method has several good properties that we use. The first and obvious
property is that the same signal has the same level string each time for the unchanged $w$. Besides this, small changes and
oscillations in the signal will eventually stay at the same level most of the time.
We mentioned that the signal consists of some events that are represented by a longer signal that is around the same level.
With this method, we hope that one event remains in the one window. Subsequent
contraction of the level string should cause that one event is in most cases represented by
one level.

Another good factor is that with the change of the parameter $w$ we can change how specific
and informative our method is. With the $w=2$ we can not say a lot about the two signals
that have level string equal. With bigger $w$ we can miss the two same signals
if there is a little noise present.

TODO: add two images with 3-5 levels and 12+ levels. Show that the oscillations are
out of a window for a higher number of levels. Then show that for a lower number of levels
two "events" are quite distinct but are on the same level.

\section{Simulating Squiggles from Reference}

We already stated that we need to obtain some reference squiggle that we
can sample and use to later split squiggles between that we are interested in and not
interested in. For this purpose, we need to take the reference in the form of a DNA sequence
and turn it into a signal. This simulation of the signal is based on the 6-mer kmer model.
So for every 6-mer in our reference DNA, we will generate signal equal to the mean
of the gaussian distribution for that particular 6-mer.

This simulation does not take into account that the signal from one nucleotide is
measured several times. We already estimated that single nucleotide is on average
measured about 10 times so we will duplicate every entry in this simulated signal
10 times. We want to see how this signal compares to real signal so to put these
two into perspective we normalize this signal by subtracting mean and removing
standard deviation \ref{obr:simVsReal}.

\begin{figure}
\centerline{\includegraphics[width=0.7\textwidth, height=0.3\textheight]{images/simulateRef}}
\caption[Hehe]{Simulated signal vs real signal, normalized}
\label{obr:simVsReal}
\end{figure}

Now we have a simulated reference signal. On \ref{obr:simVsReal} we can see that the simulated and real
signal are not identical. This poses problems as we want to have these signals as
similar as possible. The first thing that we want to consider is normalization. There
are two possible approaches, local and global normalization. We will try both these
approaches. One of the most wide-spread ways of doing this is to subtract mean and dividing
signal by standard deviation. The other possible solution is subtracting median value
as the frequent outliers can deform the mean.

Another visible problem is that the real squiggle is somehow contracted
in some places. This is caused by the fact that the DNA molecule is not moving
through the nanopore at a constant speed.

As there are differences between simulated and real signal, there are two possible
ideas on how to overcome these differences. One is to make the simulated signal more like
the real signal. The first thing that we could do would be adding some Gaussian noise
that creates a more noisy signal. There are also some more advanced simulators of
signal such as DeepSimulator. We want to go the other way and make the real signal
less noisy so we will now address the most important differences between simulated and real signal.


TODO: Add citation for DeepSimulator
% https://github.com/lykaust15/DeepSimulator

\subsection{Oscillations}

We already saw in \ref{obr:simVsReal} that one of the biggest differences between simulated and
real signal is the noise which can be seen as small oscillations of the signal. This is
one of the reasons that we use windows - so small oscillations within the window
are dealt with. We can also pretty easily adapt the number of windows as we saw in TODO.
However, when our method is used without any modifications it
produces much longer level string for real signal than for simulated signal.
The problem arises when the oscillations are located on the borders of the windows.
This is something that unfortunately happens too often.
There are several possibilities to deal with this behavior not affecting speed too
much. We tried several smoothing techniques to reduce noise. From simple as the
moving average and the moving median to some more complicated as a Kalman filter.
For now, we will stick to faster solutions and choose one of the simpler techniques.

\subsection{Signal continuity}

The next difference that was visible from the comparison of real signal level string
and the simulated signal level string is that the real signal is more continuous instead
of a simulated signal that jumps. Again, we can use the moving average. Instead of
the moving median, it has a nice advantage that it causes signal not to jump from
one level to another but rather move continuously.

TODO: Add image of signal smoothed by moving average and moving median.

\section{Identification of Reads Based on Shared k-mers}

We will now look at how the simulated reference signal compares to the raw read
corresponding to this reference and random read from the other sequencing.

\section{Experimental evaluation}


\subsection{Level string alignment}

To see how the reference level string and read level string compare to each other
we will align them. The input to the alignment algorithm are two strings $s_1$, $s_2$:
$s_1=a_1a_2\cdots a_n$, $s_2=b_1b_2\cdots b_m$. The alignment are strings
$s_3$, $s_4$: $s_3 = c_1c_2c_3\cdots c_k$, $s_4 = d_1d_2d_3\cdots d_k$
such that if $c_i \neq d_i$ then $c_i = - \lor d_i = -$ and also 

Example of allignment of $s_1 = AAACTGC$ and $s_2 = AACGTC$ is:

\begin{center}
$s_3 = $AAAC\,-\,TGC\\
$s_4 = $AA\,-\,CGT\,-\,C
\end{center}

There are a lot of alignments of the two particular strings. Most of the time we
are interested in some particular alignment. For this purpose, we need to have a
scoring system. We can, for example, minimize the number of dashes.

Using the alignment defined this way we can see how two-level strings compare to
each other and how similar they are.

Finding the best alignment is not trivial. We use a dynamic programming for
solving this problem. Let $T[i][j]$ be the best allignment of string $s_{1_i} = a_1a_2\cdots a_i$
and $s_{2_j} = b_1b_2\cdots b_j$. We will for simplification say that $\forall i, j: T[i][0] = i, T[0][j] = j$.

\[
T[i][j] = \bigl.
  \begin{cases}
    0, & \forall i,j : i\leq 0 \lor j\leq 0 \\
    max(T[i-1][j-1] + 1, T[i-1][j], T[i][j-1]) & \text{only if} \, a_i = b_i \\ 
    max(T[i-1][j], T[i][j-1]) & \text{otherwise}
  \end{cases}
\]
