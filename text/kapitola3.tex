\chapter{Proposed method}

\label{kap:proposedMethod} % id kapitoly pre prikaz ref

In this chapter, we will present our proposed solution.

\section{Method overview}

We want to create a fast yet accurate solution. Idea of our solution is to somehow
discretize the simulated signal of reference sequence and then be able to decide
during the process of sequencing if the DNA molecule currently passing through the
pore is from this reference signal. In our current setup we are not very limited
by the preprocessing time but we need to decide very fast once we are sequencing
the DNA. This puts some speed limitation on our discretizing process during the
processing of the DNA.

\section{Obtaining reference signal}

We already stated that we need to obtain some reference signal that we
can sample and use to later split reads between that we are interested and not
interested in. For this purpose, we need to take the reference in the form of DNA sequence
and turn it into a signal. We will use the nanopore data
variant caller (nadavca) which is a tool that can take a DNA sequence as an input
and produce a simulated signal. This simulation does not take into account that
the signal from one nucleotid is measured several times. We already estimated
that one nucleotid is in average measured about 10 times so we will duplicate every
entry in this simulated signal 10 times. We want to see how this signal compares
to real signal so to put these two into perspective we normalize this signal by
subtracting mean and removing standard deviation \ref{obr:simVsReal}.

\begin{figure}
\centerline{\includegraphics[width=0.7\textwidth, height=0.3\textheight]{images/simulateRef}}
\caption[Hehe]{Simulated signal vs real signal}
\label{obr:simVsReal}
\end{figure}

% https://github.com/lykaust15/DeepSimulator
There are some more advanced simulators of signal such as DeepSimulator but we
wanted to begin with the simplest possible solution.

\section{Real and simulated signal comparison}

Now we have a simulated reference signal. We can see that the simulated and real
signal are not identical in \ref{obr:simVsReal}. This poses problems as we want to have these signals as
similar as possible. The first thing that we want to consider is normalization. There
are two possible approaches, local and global normalization. We will try both this
approaches. One of the most wide-spread way of doing this is to subtract mean and dividing
signal by standard deviation. The other possible solution is subtracting median value
as the frequent outliers can deform the mean. 

The another visible problem is that real signal is somehow contracted
on some places. This is caused by the fact that the DNA molecule is not moving
through the pore at a constant speed. 

\section{Core idea}

It's good for us to represent signal visually as a graph. To certain extent we
can see that two signals are similar. However, we need to do this fast and on a very
long signals. The main goal is to be able to find the signal in the reference
signal if it is present. This is only possible if we somehow represent the signal
in a more compact form in some searchable way. What we will do is that we will
represent the signal as a string. What we will do is that we split the signal vertically
into several windows in a such manner that everything between minimal and maximal
signal is in some window. These windows are for now of constant length and does not
overlap. More formally, let $a_i$ be the signal at the time $i$, $m$ the number of
vertical windows. Now, when we have some signal $s$ of length $n$, $s=a_1a_2a_3\cdots a_n$. Besides this, we have some
values $min_a$, $max_a$ that $\forall a_i: min_a \leq a_i \leq max_a$. We say that signal $a_i$ is in $j$-th
window if it holds that:

\begin{center}
$min_a + j\cdot \frac{max_a-min_a}{m} \leq a_i < min_a + (j+1)\cdot \frac{max_a-min_a}{m}$
\end{center}
