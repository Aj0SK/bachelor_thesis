\chapter{Proposed method}

\label{kap:proposedMethod} % id kapitoly pre prikaz ref

In this chapter, we will present our proposed solution.

\section{Obtaining reference signal}

As we want to work with a signal, we need to obtain some reference signal that we
can sample and use to later split reads between that we are interested and not
interested in. For this purpose, we need to take the reference in the form of DNA sequence
and turn it into a signal. For some DNA sequence we can use the nanopore data
variant caller (nadavca) which is a tool that can take a DNA sequence as an input
and produce a simulated signal. This simulation does not take into account that
the signal from one nucleotid is measured several times. We already estimated
that one nucleotid is in average measured about 10 times so we need to duplicate every
entry in this simulated signal 10 times. We want to see how this signal compares
to real signal so to put these two into perspective we normalize this signal by
subtracting mean and removing standard deviation \ref{obr:simVsReal}.

\begin{figure}
\centerline{\includegraphics[width=0.7\textwidth, height=0.3\textheight]{images/simulateRef}}
\caption[Hehe]{Simulated signal vs real signal}
\label{obr:simVsReal}
\end{figure}

% https://github.com/lykaust15/DeepSimulator
There are some more advanced simulators of signal such as DeepSimulator but we
wanted to begin with the simplest possible solution.

\section{Real and simulated signal comparison}

Now we have a simulated reference signal. We can see that the simulated and real
signal are not identical in \ref{obr:simVsReal}. This poses problems as we want to have these signals as
similar as possible. The first thing that we want to consider is normalization. There
are two possible approaches, local and global normalization. We will try both this
approaches. One of the most wide-spread way of doing this is to subtract mean and dividing
signal by standard deviation. The other possible solution is subtracting median value
as the frequent outliers can deform the mean. 

The another visible problem is that real signal is somehow contracted
on some places. This is caused by the fact that the DNA molecule is not moving
through the pore at a constant speed. 
