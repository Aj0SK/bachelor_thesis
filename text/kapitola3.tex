\chapter{Towards Building a Squiggle Index}

\label{kap:methAdjust} % id kapitoly pre prikaz ref

In this chapter we will look at the main challenges of our method and the ways
we can overcome them.

\section{Section 1}

So now, we can transform the reference DNA sequence to signal using our simulation
process, choose the size of the window - $w$, create the level string of the reference.
Then, when the real signal from the read arrives, we will perform the same process
as with the simulated reference signal. Now we have one reference level string and
one which we want to find in it. We see that this can be better formulated as an
algorithmic problem. However, there is a very small probability that the level string
of the longer read would match exactly to some area in the reference level string.
Instead of looking for the exact match in the reference we will cut the reference
level string into overlapping subsequences of length $k$ and put them into a hashtable.
Hash table is a data structure that allows insertion and search of an element in
amortized time complexity $O(1)$. When we have this hashtable ready after the
preprocessing we can start with the actual read processing. For every read, we will
build the level string from the raw signal. Then, we will cut it into the overlapping
subsequences of the same length $k$ and see how many of them can be found in our
prepared hashtable. We call all the subsequences that are present in the hashtable
hits. Our initial assumption is that the number of hits will be considerably higher
for the read belonging to the reference.

We would want to now see how our discretization works and how similar are two level
strings of reference and the real signal that corresponds to this reference. We are
also interested in how some random read level string with our reference level string.

\subsection{Aligning the squiggle level string}

Before trying to optimize for the speed it would be good to know if our method
has at least theoretical possibility to work. For this purpose, we will try to
find our query level string in the reference level string. We already know that
it does not make sense to look for the exact match so we will need some string searching
technique that can deal with the small errors. There are more algorithms that are able
to do this. We decided to tweak a Minimap2 \cite{li2018minimap2} algorithm we already
talked about in TODO. What we will do is that we choose the number of levels $w=4$.
This will cause that the reference and query level strings will both consist of
the characters 'a', 'b', 'c', 'd'. When we will substitute 'a', 'b', 'c', 'd' by
'A', 'C', 'G', 'T' subsequently, we will obtain the manipulated level strings that
are represented using the DNA sequences. Now, we can use the Minimap2 algorithm for
finding our manipulated query level string in the manipulated reference level string.
Of course there is small catch and it is the fact that we need to ignore the hits
on the reverse strand as the reverse strand does not carry the same information
as in the real DNA sequence.

